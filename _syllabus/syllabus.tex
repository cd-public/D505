\documentclass[11pt,]{article}
\usepackage[margin=1in]{geometry}
\newcommand*{\authorfont}{\fontfamily{phv}\selectfont}
\usepackage[]{mathpazo}
\usepackage{abstract}
\renewcommand{\abstractname}{}    % clear the title
\renewcommand{\absnamepos}{empty} % originally center
\newcommand{\blankline}{\quad\pagebreak[2]}

\providecommand{\tightlist}{%
  \setlength{\itemsep}{0pt}\setlength{\parskip}{0pt}} 
\usepackage{longtable,booktabs}

\usepackage{parskip}
\usepackage{titlesec}
\titlespacing\section{0pt}{12pt plus 4pt minus 2pt}{6pt plus 2pt minus 2pt}
\titlespacing\subsection{0pt}{12pt plus 4pt minus 2pt}{6pt plus 2pt minus 2pt}

\titleformat*{\subsubsection}{\normalsize\itshape}

\usepackage{titling}
\setlength{\droptitle}{-.25cm}

%\setlength{\parindent}{0pt}
%\setlength{\parskip}{6pt plus 2pt minus 1pt}
%\setlength{\emergencystretch}{3em}  % prevent overfull lines 

\usepackage[T1]{fontenc}
\usepackage[utf8]{inputenc}

\usepackage{fancyhdr}
\pagestyle{fancy}
\usepackage{lastpage}
\renewcommand{\headrulewidth}{0.3pt}
\renewcommand{\footrulewidth}{0.0pt} 
\lhead{}
\chead{}
\rhead{\footnotesize DATA 505: Applied Machine Learning -- Spring 2021}
\lfoot{}
\cfoot{\small \thepage/\pageref*{LastPage}}
\rfoot{}

\fancypagestyle{firststyle}
{
\renewcommand{\headrulewidth}{0pt}%
   \fancyhf{}
   \fancyfoot[C]{\small \thepage/\pageref*{LastPage}}
}

%\def\labelitemi{--}
%\usepackage{enumitem}
%\setitemize[0]{leftmargin=25pt}
%\setenumerate[0]{leftmargin=25pt}




\makeatletter
\@ifpackageloaded{hyperref}{}{%
\ifxetex
  \usepackage[setpagesize=false, % page size defined by xetex
              unicode=false, % unicode breaks when used with xetex
              xetex]{hyperref}
\else
  \usepackage[unicode=true]{hyperref}
\fi
}
\@ifpackageloaded{color}{
    \PassOptionsToPackage{usenames,dvipsnames}{color}
}{%
    \usepackage[usenames,dvipsnames]{color}
}
\makeatother
\hypersetup{breaklinks=true,
            bookmarks=true,
            pdfauthor={ ()},
             pdfkeywords = {},  
            pdftitle={DATA 505: Applied Machine Learning},
            colorlinks=true,
            citecolor=blue,
            urlcolor=blue,
            linkcolor=magenta,
            pdfborder={0 0 0}}
\urlstyle{same}  % don't use monospace font for urls


\setcounter{secnumdepth}{0}

\usepackage{longtable}




\usepackage{setspace}

\title{DATA 505: Applied Machine Learning}
\author{Jameson Watts, Ph.D.}
\date{Spring 2021}


\begin{document}  

		\maketitle
		
	
		\thispagestyle{firststyle}

%	\thispagestyle{empty}


	\noindent \begin{tabular*}{\textwidth}{ @{\extracolsep{\fill}} lr @{\extracolsep{\fill}}}


E-mail: \texttt{\href{mailto:jwatts@willamette.edu}{\nolinkurl{jwatts@willamette.edu}}} & Web: \href{http://jamesonwatts.com}{\tt jamesonwatts.com}\\
Office Hours: \href{https://calendar.google.com/calendar/u/0/selfsched?sstoken=UUltRXJtZHQwTXlHfGRlZmF1bHR8YzBkZmIzM2IwZDg4ODhiNDc0NTgzZTAwOGU2YTI3ZDk}{Click
Here} for Appointment  &  Class Hours: M 06:00-09:50 p.m.\\
Office: 209 Mudd Building  & Class Room: \emph{online}\\
	&  \\
	\hline
	\end{tabular*}
	
\vspace{2mm}
	


\hypertarget{course-description}{%
\section{Course Description}\label{course-description}}

Machine learning is becoming a core component of many modern
organizational processes. It is a growing field at the intersection of
computer science and statistics focused on finding patterns in data.
Prominent applications include personalized recommendations, image
processing and speech recognition. This course will focus on the
application of existing machine learning libraries to practical problems
faced by organizations. Through lectures, cases and programming
projects, students will learn how to use machine learning to solve real
world problems, run evaluations and interpret their results.

\hypertarget{course-format}{%
\section{Course Format}\label{course-format}}

This course employs various methods, including formal presentations by
the instructor, case discussions, simulations, and in-class
activities---the approach used depends largely on the class material for
a given week. Active participation is paramount to your success in this
course. Students are expected to question, challenge, or clarify the
material as it is being presented, and to discuss issues/questions
raised by your colleagues and/or the instructor.

\hypertarget{course-objectives}{%
\section{Course Objectives}\label{course-objectives}}

\begin{enumerate}
\def\labelenumi{\arabic{enumi}.}
\tightlist
\item
  Creatively engineer new features to help with model performance
\item
  Identify and correct for sample bias, association bias, and leakage
\item
  Implement common machine learning algorithms and interpret the results
\item
  Use an application interface to run a deep learning model
\item
  Effectively communicate model results in writing and in person
\end{enumerate}

\hypertarget{course-materials}{%
\section{Course Materials}\label{course-materials}}

\begin{itemize}
\tightlist
\item
  Base R, \href{https://cran.r-project.org/}{Install from here}
\item
  Latest Version of RStudio,
  \href{https://www.rstudio.com/products/rstudio/download}{Install from
  here}
\item
  Various free resources (links are in the schedule on the syllabus)
\item
  Willamette RStudio Server (tentative):
  \href{https://rstudio-ds.willamette.edu/auth-sign-in}{Click here}
\end{itemize}

\hypertarget{graded-items}
\item
  Midterm Exam: \textbf{25\%}
\item
  Model Performance (x3): \textbf{30\%}
\item
  Final Exam: \textbf{10\%}
\item
  Group Presentations: \textbf{10\%}
\end{itemize}

\hypertarget{grade-distribution}{%
\section{Grade Distribution}\label{grade-distribution}}

\begin{longtable}[]{@{}ll@{}}
\toprule
Percentage & Grade\tabularnewline
\midrule
\endhead
\textgreater{} 95.00 & A\tabularnewline
90.00 - 94.99 & A-\tabularnewline
85.00 - 89.99 & B+\tabularnewline
80.00 - 84.99 & B\tabularnewline
75.00 - 79.99 & B-\tabularnewline
60.00 - 74.99 & C\tabularnewline
\textless{} 60.00 & F\tabularnewline
\bottomrule
\end{longtable}

\hypertarget{assignments}{%
\section{Assignments}\label{assignments}}

\begin{itemize}

            
    \item \textbf{Homework Assignments (25\%):} You will be assigned 6 individual homework assignments over the course of the semester. These will typically involve writing some code and "knitting" a file to pdf or html in order to turn it in. Your lowest score will be dropped.
    
    \item \textbf{Exams (35\%):} We will have one midterm exam worth 25\% of your grade. This exam will involve writing code to complete various machine learning tasks. You can expect to be given a dataset and a series of questions to answer using the skills developed during the course. You will have three hours to complete this exam. Your final exam will be much shorter and largely conceptual in nature. You will have 90 minutes to complete the final. Exams are open everything (book, notes, internet), EXCEPT communication with others.

    \item \textbf{Machine Learning Models and Presentation (40\%):} Over the course of the semester you will be working in a group tasked with creating several models that classify and/or predict some feature of our wine dataset. For each modeling assignment, I will provide the training and test data used to develop and measure your model's performance. Your group will have three opportunities to build a machine learning model. At each of these opportunities, the relative performance of your model determines your grade---the top group earns 20 points, 19 for second place 18 points for third place, 17 points for fourth, and 16 points for 5th. During our last class you will present your model results as though you are speaking to the managers of a large winery. Details and expectations will be clarified in class.
\end{itemize}

\newpage

\hypertarget{course-policies}{%
\section{Course Policies}\label{course-policies}}

Detailed policies for this course are below. Basically, don't cheat and
try to learn stuff.

\begin{itemize}
\tightlist
\item
  Video must be turned on in the Zoom call and names displayed.
\item
  No private chat
\item
  No late assignments except in \textbf{very rare} cases of personal or
  family emergency.
\item
  Students with disabilities who require accommodation should notify me
  of the nature of accommodation you require in the first week of class.
  Additional support is available from the Willamette University
  \href{www.willamette.edu/dept/disability}{Accessible Education
  Services Office}, telephone 503-370-6471.
\item
  Students are responsible for all missed work, regardless of the reason
  for absence. It is also the absentee's responsibility to get all
  missing notes or materials.
\item
  Every student is expected at all times to abide by the Willamette
  University
  \href{http://www.willamette.edu/mba/about/honorcode}{Atkinson Graduate
  School of Management Honor Code}.

  \item

  You must also abide by the Application to Academic Honesty as detailed
  in the
  \href{http://www.willamette.edu/mba/students/student-handbook}{current
  student handbook}.
\end{itemize}

\newpage

\hypertarget{course-schedule}{%
\section{Course Schedule}\label{course-schedule}}

\hypertarget{week-01-0111-machine-learning-overview}{%
\subsection{Week 01, 01/11: Machine Learning
Overview}\label{week-01-0111-machine-learning-overview}}

\begin{itemize}
\tightlist
\item
  Read the syllabus
\item
  Read Chapter 1 of
  \href{https://srdas.github.io/MLBook/DataScience.html}{this book}
\item
  Dust off your tidyverse skills
\end{itemize}

\hypertarget{week-02-0118}{%
\subsection{\texorpdfstring{Week 02, 01/18:
\textcolor{Plum}{No Class: MLK}}{Week 02, 01/18: }}\label{week-02-0118}}

\hypertarget{week-03-0125-feature-engineering-i-variable-selection}{%
\subsection{Week 03, 01/25: Feature Engineering I \& Variable
Selection}\label{week-03-0125-feature-engineering-i-variable-selection}}

\begin{itemize}
\tightlist
\item
  Read the Preface and Ch. 1 of \href{http://www.feat.engineering/}{this
  book}
\item
  \emph{\textcolor{Bittersweet}{Homework 1 due}}
\end{itemize}

\hypertarget{section}{%
\subsection{\texorpdfstring{\emph{\textcolor{OliveGreen}{---Supervised Learning---}}}{}}\label{section}}

\hypertarget{week-04-0201-k-nearest-neighbors}{%
\subsection{Week 04, 02/01: K-Nearest
Neighbors}\label{week-04-0201-k-nearest-neighbors}}

\begin{itemize}
\tightlist
\item
  \emph{\textcolor{Bittersweet}{Homework 2 due}}
\end{itemize}

\hypertarget{week-05-0208-naive-bayes}{%
\subsection{Week 05, 02/08: Naive
Bayes}\label{week-05-0208-naive-bayes}}

\begin{itemize}
\tightlist
\item
  \emph{\textcolor{Bittersweet}{Homework 3 due}}
\end{itemize}

\hypertarget{week-06-0215-logistic-regression}{%
\subsection{Week 06, 02/15: Logistic
Regression}\label{week-06-0215-logistic-regression}}

\begin{itemize}
\tightlist
\item
  \emph{\textcolor{Bittersweet}{Homework 4 due}}
\end{itemize}

\hypertarget{week-07-0222-decision-trees}{%
\subsection{Week 07, 02/22: Decision
Trees}\label{week-07-0222-decision-trees}}

\begin{itemize}
\tightlist
\item
  \emph{\textcolor{Bittersweet}{Homework 5 due}}
\end{itemize}

\hypertarget{week-08-0301-boosting-and-bagging}{%
\subsection{Week 08, 03/01: Boosting and
Bagging}\label{week-08-0301-boosting-and-bagging}}

\hypertarget{week-09-0308-review}{%
\subsection{Week 09, 03/08: Review}\label{week-09-0308-review}}

\begin{itemize}
\tightlist
\item
  \emph{\textcolor{Bittersweet}{First group model due}}
\end{itemize}

\hypertarget{week-10-0315-mid-term-exam}{%
\subsection{Week 10, 03/15: Mid-term
Exam}\label{week-10-0315-mid-term-exam}}

\hypertarget{week-11-0322}{%
\subsection{\texorpdfstring{Week 11, 03/22:
\textcolor{Plum}{No Class: Spring Break}}{Week 11, 03/22: }}\label{week-11-0322}}

\hypertarget{section-1}{%
\subsection{\texorpdfstring{\emph{\textcolor{OliveGreen}{---Unsupervised Learning---}}}{}}\label{section-1}}

\hypertarget{week-12-0329-feature-engineering-ii-dimensionality-reduction}{%
\subsection{Week 12, 03/29: Feature Engineering II \& Dimensionality
Reduction}\label{week-12-0329-feature-engineering-ii-dimensionality-reduction}}

\hypertarget{week-13-0405-clustering}{%
\subsection{Week 13, 04/05: Clustering}\label{week-13-0405-clustering}}

\begin{itemize}
\tightlist
\item
  \emph{\textcolor{Bittersweet}{Second group model due}}
\end{itemize}

\hypertarget{week-14-0412-the-tidy-models-package}{%
\subsection{Week 14, 04/12: The Tidy Models
Package}\label{week-14-0412-the-tidy-models-package}}

\hypertarget{week-15-0419-deep-learning-and-ensembles}{%
\subsection{Week 15, 04/19: Deep Learning and
Ensembles}\label{week-15-0419-deep-learning-and-ensembles}}

\begin{itemize}
\tightlist
\item
  \emph{\textcolor{Bittersweet}{Final group model due}}
\end{itemize}

\hypertarget{week-16-0426-final-exam-and-team-presentations}{%
\subsection{Week 16, 04/26: Final Exam and Team
Presentations}\label{week-16-0426-final-exam-and-team-presentations}}




\end{document}

\makeatletter
\def\@maketitle{%
  \newpage
%  \null
%  \vskip 2em%
%  \begin{center}%
  \let \footnote \thanks
    {\fontsize{18}{20}\selectfont\raggedright  \setlength{\parindent}{0pt} \@title \par}%
}
%\fi
\makeatother
